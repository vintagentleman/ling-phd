\section{Чат-боты}

\frame{\tableofcontents[currentsection]}

\begin{frame}
    \frametitle{Вехи}

    \begin{enumerate}
        \item<1-> ELIZA (Дж.~Вейценбаум, 1966) \begin{itemize}
            \item пародия на психотерапевта
            \item шаблоны запросов и правила их трансформации в ответы
            \item управление памятью
        \end{itemize}

        \item<2-> SHRDLU (Т.~Виноград, 1968--1970) \begin{itemize}
            \item бот функционирует в мире объемных геометрических фигур
            \item умеет перемещать объекты, запоминает их названия, величину и цвет
            \item умеет разрешать кореференцию
        \end{itemize}

        \item<3-> PARRY (1971) \begin{itemize}
            \item бот для изучения шизофрении
            \item управление уровнем страха и раздражения
            \item первый бот, формально прошедший тест Тьюринга
        \end{itemize}
    \end{enumerate}
\end{frame}

\begin{frame}
    \frametitle{Подход на правилах}

    \begin{itemize}
        \item шаблоны для обработки запросов (регулярные выражения, etc.)
        \item правила ранжирования совпадающих шаблонов
        \item заготовленные ответы
    \end{itemize}
\end{frame}

\begin{frame}[fragile]
    \frametitle{Подход на правилах сегодня}

    \begin{Verbatim}[fontsize=\scriptsize, gobble=8]
        state: Peterburg
          q!: * (~печаль|~тлен|~боль|~грусть|~скорбь|~страдание|~страдать|~плакать) *
          random:
            a: В этом мире есть боль и печаль... Но радости и любви в нём гораздо больше!
            a: Только тот узнает счастье, кто печаль перенесёт.
            a: Радости нет без печали.
    \end{Verbatim}
\end{frame}

\begin{frame}
    \frametitle{Корпусный подход}

    \begin{block}{Дано}
        \begin{itemize}
            \item Корпус обучающих диалогов, содержащих пары <<вопрос~--- ответ>>
            \item Контекст предшествующего диалога с пользователем
            \item Текущий запрос пользователя
        \end{itemize}
    \end{block}

    \begin{alertblock}{Найти}
        Ответ на запрос, уместный в контексте диалога
    \end{alertblock}
\end{frame}

\begin{frame}
    \frametitle{Подходы к нахождению ответа}

    \begin{columns}
        \column{.5\textwidth}
        \begin{block}{Response by retrieval}
            \begin{itemize}
                \item Вытаскиваем ответ на похожий запрос из корпуса
                \item Похожесть вычисляем на основе какой-либо метрики
                \item Или используем bi-encoder
            \end{itemize}
        \end{block}

        \column{.5\textwidth}
        \begin{block}{Response by generation}
            \begin{itemize}
                \item Используем encoder-decoder, сразу генерируем ответ
            \end{itemize}
        \end{block}
    \end{columns}
\end{frame}
