\documentclass[xetex, aspectratio=169, russian]{beamer}
\usecolortheme{beaver}
\beamertemplatenavigationsymbolsempty
\setbeamertemplate{footline}[frame number]

\usepackage{fontspec}
\usepackage{polyglossia}
\usepackage[autostyle]{csquotes}
\setmainlanguage[babelshorthands]{russian}
\setotherlanguage{english}

\setmainfont{Liberation Serif}
\newfontfamily{\cyrillicfont}{Liberation Serif}
\setsansfont{Liberation Sans}
\newfontfamily{\cyrillicfontsf}{Liberation Sans}
\setmonofont{Liberation Mono}
\newfontfamily{\cyrillicfonttt}{Liberation Mono}

\usepackage{booktabs}
\usepackage{multirow}
\usepackage{makecell}
\usepackage{ltablex}
\usepackage{paralist}
\keepXColumns
\renewcommand\theadfont{\normalsize}

\usepackage{fvextra}
\usepackage{xcolor}
\usepackage{textcomp}
\usepackage{graphicx}
\usepackage{tikz}
\usepackage[edges]{forest}
\graphicspath{{../img/}}

\usepackage{mathtools}
\usepackage{amssymb}

% Библиография
\input{_biblatex}

\title[]{Автоматические диалоговые системы}

\begin{document}


\frame{\titlepage}

\section{Вопросы терминологии}

\frame{\tableofcontents[currentsection]}

\begin{frame}
    \frametitle{Хрестоматийное определение}
    \framesubtitle{\autocite[7]{zakharov_bogdanova:2011}}

    \begin{block}{Корпусная лингвистика}
        Раздел компьютерной лингвистики, занимающийся разработкой общих принципов построения и использования лингвистических корпусов (корпусов текстов) с применением компьютерных технологий.
    \end{block}

    \vfill

    \begin{block}{Корпус}
        Большой, представленный в машиночитаемом виде, унифицированный, структурированный, размеченный, филологически компетентный массив языковых данных, предназначенный для решения конкретных лингвистических задач.
    \end{block}
\end{frame}

\begin{frame}
    \frametitle{Определение пожестче}
    \framesubtitle{\autocite[22--23, 56]{stefanowitsch:2020}}

    \begin{block}{Corpus linguistics}
        The investigation of linguistic research questions that have been framed in terms of the conditional distribution of linguistic phenomena in a linguistic corpus.
    \end{block}

    \vfill

    \begin{block}{Linguistic corpus}
        A collection of samples of language use with the following properties: \begin{itemize}
            \item the instances of language use contained in it are \textit{authentic};
            \item the collection is \textit{representative} of the language or language variety under investigation;
            \item the collection is \textit{large}.
        \end{itemize}
    \end{block}
\end{frame}

\begin{frame}
    \frametitle{Corpus-based linguistics?}
    \framesubtitle{\autocite{weisser:2016}}

    \begin{quote}
        [\ldots] the term corpus linguistics, while shorter and more popular, tends to give the impression that it is a branch of linguistics rather than just a methodology which can be applied to any existing branch of linguistics. \underline{Our interest should be in language, not corpora per se.}
    \end{quote}
\end{frame}

\begin{frame}
    \frametitle{Интерпретация как основа создания корпусов}
    \framesubtitle{\autocite[6--12]{brown_yule:1983}}

    \begin{quote}
        It must be further emphasised that, however objective the notion of `text' may appear as we have defined it (`the verbal record of a communicative act'), the perception and interpretation of each text is essentially subjective.
    \end{quote}

    \begin{quote}
        In discussing texts we idealise away from this variability of the experiencing of the text and assume what Schutz has called `the reciprocity of perspective'. [\ldots]
    \end{quote}
\end{frame}

% TODO

\section{Разметка корпусов}

\frame{\tableofcontents[currentsection]}

\begin{frame}
    \frametitle{Виды разметки}

    \begin{itemize}
        \item Лингвистическая: морфология, синтаксис, семантические роли, именованные сущности
        \item Паралингвистическая: особенности почерка, иллюстрации к рукописи; просодика, жесты
        \item Экстралингвистическая: характеристики автора, жанр, историческая эпоха и место создания
    \end{itemize}
\end{frame}

\begin{frame}
    \frametitle{Немного о стандартах}
    \framesubtitle{\url{https://xkcd.com/927}}
    \centering
    \includegraphics[width=.9\textwidth]{standards}
\end{frame}

\subsection{Форматы морфосинтаксической разметки}

\frame{\tableofcontents[currentsection, currentsubsection]}

\begin{frame}
    \frametitle{Universal Dependencies}
    content...
\end{frame}

\begin{frame}
    \frametitle{MULTEXT-East}

    \begin{itemize}
        \item Спецификация для 20~языков \autocite{erjavec:2021}
        \item Лексиконы
        \item Параллельный корпус романа <<1984>> на 12~языках
    \end{itemize}

    \vfill

    \begin{block}{Пример тегсета}
        \textit{он} \linebreak \texttt{P-3msn} \linebreak мест., тип не задан, 3~л., м.~р., ед.~ч., им.~п.
    \end{block}
\end{frame}

\begin{frame}
    \frametitle{PROIEL}
    content...
\end{frame}

\subsection{Инструменты}

\frame{\tableofcontents[currentsection, currentsubsection]}

\begin{frame}
    \frametitle{ANNIS}
    content...
\end{frame}

% TODO

\section{Источники}

\frame{\tableofcontents[currentsection]}

\begin{frame}{Источники}
    \nocite{*}
    \printbibliography
\end{frame}

\begin{frame}{}
    \centering

    \vfill
    Q\&A
    \vfill
\end{frame}

\section{Литература}

\frame{\tableofcontents[currentsection]}

\begin{frame}
    \frametitle{Литература}
    \nocite{*}
    \printbibliography
\end{frame}

\end{document}

