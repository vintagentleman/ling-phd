\section{Словообразование}

\subsection{Основные понятия}

\frame{\tableofcontents[currentsection,currentsubsection]}

\begin{frame}
    \frametitle{Словообразование}
    \framesubtitle{\autocite{les}}

    Раздел языкознания, изучающий все аспекты \begin{itemize}
        \item создания,
        \item функционирования,
        \item строения
        \item и классификации
    \end{itemize} производных и сложных слов
\end{frame}

\begin{frame}
    \frametitle{Номинация}
    \framesubtitle{\autocite[98--100]{zubova_menshikova:2014}}

    \begin{center}
        \begin{forest}
            forked edges,
            [Номинация
                [Первичная]
                [\textbf{Вторичная}
                    [ПС $+$ ПВ
                        [Словопроизводство]
                        [Словосложение]
                        [\ldots]
                    ]
                    [ПС
                        [Конверсия]
                        [Полисемия]
                    ]
                ]
            ]
        \end{forest}

        \vfill

        \onslide<2->{\textit{Le signe linguistique est arbitraire ?}}
    \end{center}
\end{frame}

\begin{frame}
    \frametitle{Производное слово}

    \textit{преподаватель:} \begin{itemize}
        \item Отсылочная часть: `преподавать' ($\sim$ тема)
        \item Формантная часть: `субъект действия' ($\sim$ рема)
    \end{itemize}

    \vfill

    \begin{block}<2->{Образование и изменение}
        производное слово $-$ производящее слово $=$ \begin{enumerate}
            \item лексическое значение? --- слово\textbf{образование}
            \item синтаксическое значение? --- слово\textbf{изменение}
            \item грамматическое значение? --- \textbf{формо}образование
        \end{enumerate}
    \end{block}

\end{frame}

\subsection{Словообразовательный анализ}

\frame{\tableofcontents[currentsection,currentsubsection]}

\begin{frame}
    \frametitle{Словообразование и морфемика}
    \framesubtitle{\autocite[16--19]{gerd:2004:morphology}}

    \begin{columns}
        \column{.5\textwidth}
        \begin{block}{Морфемный анализ}
            \begin{itemize}
                \item \textit{уч-и-тель-ств-о}
                \item \textbf{структурный тип}~--- P--R--3S--F
            \end{itemize}
        \end{block}

        \column{.5\textwidth}
        \begin{block}<2->{Словообразовательный анализ}
            \begin{itemize}
                \item \textit{учить} > \textit{учительство}
                \item {[учи]}~$+$ [тельство]
                \item \textbf{словообразовательный тип}~--- V~$+$ S
                \item V~$+$ \textit{тельство} `регулярное занятие'
            \end{itemize}
        \end{block}
    \end{columns}
\end{frame}

\begin{frame}
    \frametitle{Морфологические (аффиксальные) способы словообразования}
    \framesubtitle{\autocite[122--131]{zubova_menshikova:2014}}

    \begin{itemize}
        \item Суффиксальный: \textit{хвост}~$\rightarrow$ \textit{хвостище}
        \item Префиксальный: \textit{порядок}~$\rightarrow$ \textit{беспорядок}
        \item Постфиксальный: \textit{собирать}~$\rightarrow$ \textit{собираться}
        \item Префиксально-суффиксальный: \textit{осина}~$\rightarrow$ \textit{подосиновик}
        \item \ldots
        \item Сложение: \textit{снег} $+$ \textit{ходить}~$\rightarrow$ \textit{снегоход}
        \item Обратное словообразование: \textit{неформальный}~$\rightarrow$ \textit{неформал}
    \end{itemize}
\end{frame}

\begin{frame}
    \frametitle{Неморфологические способы}
    \framesubtitle{\autocite[131--136]{zubova_menshikova:2014}}

    \begin{itemize}
        \item Аббревиация: \textit{СПбГУ}
        \item Усечение: \textit{компьютер}~$\rightarrow$ \textit{комп}
        \item Лексико-синтаксический: \textit{дикорастущий}
        \item Морфолого-синтаксический (транспозиция): \textit{прохожий}
        \item Лексико-семантический: \textit{язык} `орган'~$\rightarrow$ `система знаков'
    \end{itemize}
\end{frame}

\begin{frame}
    \frametitle{Семантические отношения}

    \begin{block}{Полисемия}
        Adj~$+$ S~$\rightarrow$ Adj: \begin{itemize}
            \item Усиление качества: \textit{умн-ейш-ий}
            \item Ослабление качества: \textit{грязн-оват-ый}
        \end{itemize}
    \end{block}

    \begin{exampleblock}{Синонимия}
        \begin{itemize}
            \item \textit{ширь}~--- \textit{ширина}~--- \textit{широта}
            \item \textit{выбрать}~--- \textit{избрать}
        \end{itemize}
    \end{exampleblock}

    \begin{alertblock}{Омонимия}
        \begin{itemize}
            \item \textit{проводник} `проводящий материал' и `сопровождающее лицо'
            \item \textit{перечитать} `заново прочитать' и `прочитать в больших количествах'
        \end{itemize}
    \end{alertblock}
\end{frame}

\subsection{Синхрония и диахрония}

\frame{\tableofcontents[currentsection,currentsubsection]}

\begin{frame}
    \frametitle{О слове <<позволить>>}
    \framesubtitle{\url{https://vk.com/glazslov?w=wall-71343900_8367}}

    \begin{itemize}
        \item \textit{позвол-ить?}
            \onslide<2->{Отрицание смысловой и формальной связи с \textit{воля}, \textit{дозволить}~--- абсурд}
        \item \textit{поз-вол-ить?}
            \onslide<3->{\textit{поз-}~--- унификс? \textit{-з-}~--- интерфикс? И прочие интеллектуальные игры}
        \item \textit{по-звол-ить?}
            \onslide<4->{\textit{-звол-}~--- связанный корень, плюс уходим от проблемы словообразовательного значения}
        \item<5-> Синхронное словообразование ничего в действительности не объясняет, а придумано только для потехи отвлеченного ума, игры в бисер и создания касты посвященных
    \end{itemize}
\end{frame}
