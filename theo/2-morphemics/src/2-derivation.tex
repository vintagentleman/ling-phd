\section{Словообразование}
\newrefsection

\subsection{Основные понятия}

\frame{\tableofcontents[currentsection,currentsubsection]}

\begin{frame}
    \frametitle{Определение словообразования}
    content...
\end{frame}

\begin{frame}
    \frametitle{Словообразование и морфемика}
    \framesubtitle{\autocite[16--19]{gerd:2004:morphology}}

    \begin{columns}
        \column{.5\textwidth}
        \begin{block}{Морфемный анализ}
            \begin{itemize}
                \item \textit{уч-и-тель-ств-о}
                \item \textbf{структурный тип}~--- P---R---3S---F
            \end{itemize}
        \end{block}

        \column{.5\textwidth}
        \begin{block}{Словообразовательный анализ}
            \begin{itemize}
                \item \textit{учить} > \textit{учительство}
                \item {[учи]}~$+$ [тельство]
                \item \textbf{словообразовательный тип}~--- V~$+$ S
                \item V~$+$ \textit{тельство} `регулярное занятие'
            \end{itemize}
        \end{block}

    \end{columns}
\end{frame}

% TODO

\subsection{Литература}

\frame{\tableofcontents[currentsection,currentsubsection]}

\begin{frame}
    \frametitle{Литература}
    \printbibliography
\end{frame}
