\section{Морфемика}
\newrefsection

\subsection{Основные понятия}

\frame{\tableofcontents[currentsection,currentsubsection]}

\begin{frame}
    \frametitle{Понятие морфемики}
    \framesubtitle{\autocite{les}}

    \begin{block}{Морфемика}
        Раздел языкознания, изучающий типы и структуру \textbf{морфем}, их отношения друг к другу и к \textbf{слову} в целом.
    \end{block}

    \begin{block}{Предмет морфемики}
        \begin{itemize}
            \item Виды морфем
            \item Типы значений, выражаемых морфемами
            \item Парадигматика и синтагматика на морфемном уровне
            \item Принципы выделения и правила отождествления морфов
            \item<2-> Звуковые изменения при сочетании единиц, меньших, чем слово
            \item<2-> Модели фонологической структуры морфов
        \end{itemize}
    \end{block}
\end{frame}

\begin{frame}
    \frametitle{Морфология}

    \begin{exampleblock}{\autocite[30]{melchuk:1997}}
        Часть лингвистики, занимающаяся словом во всех его релевантных аспектах
    \end{exampleblock}

    \hfill

    \begin{center}
        \begin{forest}
            forked edges,
            [Морфология
                [\textbf{Морфемика}
                    [Морфонология]
                ]
                [Словообразование]
                [Форфообразование]
            ]
        \end{forest}
    \end{center}
\end{frame}

\begin{frame}
    \frametitle{Морфема}

    \begin{exampleblock}{\autocite[290--291]{baudouin:1963}}
        Не разложимый морфологический элемент языкового мышления
    \end{exampleblock}

    \begin{exampleblock}{\autocite{les}}
        Минимальный знак
    \end{exampleblock}

    \onslide<2->{
        \begin{table}
            \begin{tabularx}{\textwidth}{XXX}
                & $+$ минимальность & $-$ минимальность \\ \midrule
                $+$ значение & \textbf{Морфема} & Слово и предложение \\ \midrule
                $-$ значение & Фонема & Слог \\
            \end{tabularx}
        \end{table}
    }
\end{frame}

\begin{frame}
    \frametitle{Морф}

    \begin{exampleblock}{\autocite{bloomfield:1933}}
        A linguistic form which bears no partial phonetic-semantic resemblance to any other form
    \end{exampleblock}

    \begin{table}
        \begin{tabularx}{\textwidth}{XXX}
            Уровень & Эмическая единица & Этическая единица \\ \midrule \midrule
            \ldots & \ldots & \ldots \\ \midrule
            Морфология & Морфема & Морф \\
        \end{tabularx}
    \end{table}


    \begin{columns}
        \column{.5\textwidth}
        \begin{center}
            \onslide<2->{\hfill Бихава̄виэтэм.} \onslide<3->{\hfill $\xrightarrow{?}$}
        \end{center}

        \column{.5\textwidth}
        \onslide<3->{
            \digloss
                {Би хава̄-ви             этэ-м.}
                {я  работа\textsc{.GEN} кончать\textsc{.1SG}}
                {Я закончил работу.}
        }
   \end{columns}
\end{frame}

\begin{frame}
    \frametitle{Алломорф}
    content...
\end{frame}

\subsection{Морфемное членение}

\frame{\tableofcontents[currentsection,currentsubsection]}

\begin{frame}
    \frametitle{Этап 1}
    \framesubtitle{}

    content...
\end{frame}

% TODO

\subsection{Типы морфем}

\frame{\tableofcontents[currentsection,currentsubsection]}

\begin{frame}
    \frametitle{Структурные типы}


\end{frame}

\subsection{Морфонология}

\frame{\tableofcontents[currentsection,currentsubsection]}



\subsection{Литература}

\frame{\tableofcontents[currentsection,currentsubsection]}

\begin{frame}
    \frametitle{Литература}
    \printbibliography
\end{frame}
