\section{Морфемика}
\newrefsection

\subsection{Основные понятия}

\frame{\tableofcontents[currentsection,currentsubsection]}

\begin{frame}
    \frametitle{Понятие морфемики}
    \framesubtitle{\autocite{les}}

    \begin{block}{Морфемика}
        Раздел языкознания, изучающий типы и структуру \textbf{морфем}, их отношения друг к другу и к \textbf{слову} в целом
    \end{block}

    \begin{block}{Предмет морфемики}
        \begin{itemize}
            \item Виды морфем
            \item Типы значений, выражаемых морфемами
            \item Парадигматика и синтагматика на морфемном уровне
            \item Принципы выделения и правила отождествления морфов
            \item<2-> Звуковые изменения при сочетании единиц, меньших, чем слово
            \item<2-> Модели фонологической структуры морфов
        \end{itemize}
    \end{block}
\end{frame}

\begin{frame}
    \frametitle{Морфология}

    \begin{exampleblock}{\autocite[30]{melchuk:1997}}
        Часть лингвистики, занимающаяся словом во всех его релевантных аспектах
    \end{exampleblock}

    \hfill

    \begin{center}
        \begin{forest}
            forked edges,
            [Морфология
                [\textbf{Морфемика}
                    [Морфонология]
                ]
                [Словообразование]
                [Форфообразование]
            ]
        \end{forest}
    \end{center}
\end{frame}

\begin{frame}
    \frametitle{Морфема}

    \begin{exampleblock}{\autocite[290--291]{baudouin:1963}}
        Не разложимый морфологический элемент языкового мышления
    \end{exampleblock}

    \begin{exampleblock}{\autocite{les}}
        Минимальный знак
    \end{exampleblock}

    \onslide<2->{
        \begin{table}
            \begin{tabularx}{\textwidth}{XXX}
                & $+$ минимальность & $-$ минимальность \\ \midrule
                $+$ значение & \textbf{Морфема} & Слово и предложение \\ \midrule
                $-$ значение & Фонема & Слог \\
            \end{tabularx}
        \end{table}
    }
\end{frame}

\begin{frame}
    \frametitle{Морф}
    \framesubtitle{\autocites[161]{bloomfield:1973}{haspelmath:2020}}

    \begin{block}{Morpheme (simple form)}
        A linguistic form which bears no partial phonetic-semantic resemblance to any other form
    \end{block}

    \begin{table}
        \begin{tabularx}{\textwidth}{XXX}
            Уровень & Эмическая единица & Этическая единица \\ \midrule \midrule
            \ldots & \ldots & \ldots \\ \midrule
            Морфология & Морфема & Морф \\
        \end{tabularx}
    \end{table}

    \begin{columns}
        \column{.5\textwidth}
        \begin{center}
            \onslide<2->{\hfill эвенк.\ \textit{Бихава̄виэтэм.}} \onslide<3->{\hfill $\xrightarrow{?}$}
        \end{center}

        \column{.5\textwidth}
        \onslide<3->{
            \digloss
                {Би хава̄-ви             этэ-м.}
                {я  работа\textsc{.GEN} кончать\textsc{.1SG}}
                {Я закончил работу.}
        }
   \end{columns}
\end{frame}

\begin{frame}
    \frametitle{Алломорф}
    \framesubtitle{\autocite{gerd:2004:morph}}

    \begin{columns}
        \column{.5\textwidth}
        \begin{block}{Дескриптивизм}
            Ступень перехода от уровня варианта к инварианту
        \end{block}

        \column{.5\textwidth}
        \begin{block}<2->{Русистика}
            Алломорфы одной морфемы имеют: \begin{itemize}
                \item тождественное значение
                \item \alert{фонематическую близость}
            \end{itemize}
        \end{block}
    \end{columns}

    \vfill

    \onslide<3->{
        /z/ `PL': \begin{itemize}
            \item {<<Близкое>>:} [z] (cubs), [s] (clefs), [iz] (glasses)
            \item {<<Далекое>>:} [ən] (oxen), $[\varnothing]$ (sheep), [aɪ] (alumni), [ə] (dicta)
        \end{itemize}
    }
\end{frame}

\begin{frame}
    \frametitle{Морфонема}
    \framesubtitle{\autocite{gerd:2004:morph}}

    \begin{exampleblock}{\autocite{les}}
        Элементарная единица морфонологии, предельный элемент \textbf{означающего} морфемы
    \end{exampleblock}

    \vfill

    \begin{itemize}
        \item морфы соотносятся не с алломорфами, а с морфонемами
        \item объединяющим для морфонемы как класса эквивалентности является не <<близость>>, а фонологические правила
        \item отождествление морфов в морфонемы~--- задача сегментации, а алломорфов в морфемы~--- интерпретации
    \end{itemize}
\end{frame}

\begin{frame}
    \frametitle{Объект морфонологии}
    \framesubtitle{\autocite{les}}

    \begin{alertblock}{Узкая трактовка}
        Чередование фонем в составе морфов
    \end{alertblock}

    \begin{exampleblock}{Широкая трактовка}
        \begin{itemize}
            \item Фонологический состав морфем разных типов
            \item Преобразования морфем при формо- и словообразовании
            \item Пограничные сигналы и явления на стыке морфем
        \end{itemize}
    \end{exampleblock}
\end{frame}

\subsection{Морфемное членение}

\frame{\tableofcontents[currentsection,currentsubsection]}

\begin{frame}
    \frametitle{Этап 1}
    content...
\end{frame}

% TODO

\subsection{Классификация морфем}

\frame{\tableofcontents[currentsection,currentsubsection]}

\begin{frame}
    \frametitle{Корни и аффиксы}
    \framesubtitle{\autocite[13--14]{zubova_menshikova:2014}}

    \begin{itemize}
        \item Носители лексического \textit{vs} формо- или словообразовательного значения
        \item Открытый \textit{vs} закрытый характер множества
        \item Аффиксы регулярны в формо- и словообразовательных моделях
        \item Аффиксы легко выступают в качестве нулевой морфемы
    \end{itemize}
\end{frame}

\begin{frame}
    \frametitle{Типы аффиксов}
    \framesubtitle{\autocite[14--15]{zubova_menshikova:2014}}

    \begin{itemize}
        \item Флексия: \textit{морфем-а} \begin{itemize}
            \item Формообразовательное значение
            \item Кумулятивный характер
        \end{itemize}
        \item Префикс: \textit{ин-вариант} (в рус.\ только словоизменительное значение)
        \item Суффикс: \textit{знач-ени|j|-е}
        \item Постфикс: \textit{-ся;} \textit{-то}, \textit{-либо}, \textit{-нибудь}
    \end{itemize}
\end{frame}

\begin{frame}
    \frametitle{Нелинейные аффиксы}
    \framesubtitle{\autocite[91--97]{plungyan:2003}}

    \begin{itemize}
        \item Инфикс \begin{itemize}
            \item тагальск. \textit{kagat} `укус'~--- \textit{k-um-agat} `укусил'
            \item ст.-сл.\ \textit{сѣсти} < прасл.\ \textit{*sěs-}~--- \textit{сѧдѫ} < \textit{*sęd-} < \textit{*sěnd}
        \end{itemize}
        \item<2-> Трансфикс (на примере араб.\ \textit{DXL}) \begin{itemize}
            \item \textit{daxal-a} `он вошел'
            \item \textit{ya-dxul-u} `он входил'
            \item \textit{yu-daxxal-u} `его вводят'
            \item \textit{dāxil-un} `чужой; пришелец; гость'
        \end{itemize}
        \item<3-> Циркумфикс: \begin{itemize}
            \item нем.\ \textit{ge-~$+$ -t}
            \item рус.\ \textit{до-~$+$ -ся}
        \end{itemize}
    \end{itemize}
\end{frame}

\begin{frame}
    \frametitle{Экзотика}
    \framesubtitle{\autocite[91--97]{plungyan:2003}, \autocite[15, 102]{zubova_menshikova:2014}}

    \begin{block}{Морфоиды}
        \begin{itemize}
            \item Префиксоиды: \textit{мини-}, \textit{квази-}
            \item Суффиксоиды: \textit{-вед}, \textit{-фоб}
        \end{itemize}
    \end{block}

    \begin{block}<2->{Субморфы}
        Интерфиксы: \begin{enumerate}
            \item Соединительные сегменты: \textit{пар-о-ход}
            \item Асемантические сегменты: \textit{америк-ан-ский} (алломорфирование?)
        \end{enumerate}
    \end{block}
\end{frame}

\begin{frame}
    \frametitle{Агглютинация и фузия}
    % TODO
\end{frame}

\begin{frame}
    \frametitle{Аддитивная модель морфологии}
    % TODO
\end{frame}

\subsection{Литература}

\frame{\tableofcontents[currentsection,currentsubsection]}

\begin{frame}[allowframebreaks]
    \frametitle{Литература}
    \printbibliography
\end{frame}
