\section{Словосочетание}

\subsection{Синтаксические единицы}
\frame{\tableofcontents[currentsection,currentsubsection]}

\begin{frame}
  \frametitle{Текст}

  Свойства: \begin{enumerate}
    \item Законченность (оценка со стороны говорящего)
    \item Целостность (со стороны слушающего)
  \end{enumerate}

  \vfill

  $\rightarrow$ Сверхфразовое единство \\
  $\rightarrow$ Простое предложение \\
  $\rightarrow$ Словосочетание \\
  $\rightarrow$ Слово
\end{frame}

\begin{frame}
  \frametitle{Особенности синтаксических единиц}

  \begin{enumerate}
    \item Потенциальная бесконечность множества
    \item Способность к неограниченному усложнению \begin{itemize}
      \item Добавление зависимого слова к одному из компонентов: \textit{муж друга сына учителя\ldots}
      \item Подчинение всего выражения новой синтаксической вершине: \textit{Вот два петуха, которые будят того пастуха, который\ldots}
    \end{itemize}
  \end{enumerate}
\end{frame}

\begin{frame}
  \frametitle{Словосочетание}

  Синтаксическая конструкция, образующаяся на основе подчинительных связей. Обязательно есть главный компонент (вершина) и 1+ зависимых

  \vfill

  \begin{block}<2->{Критерии}
    \begin{enumerate}
      \item Грамматический~--- отсутствие предикативности
      \item Функциональный~--- расчлененное выражение понятия
      \item Структурный~--- конструкция из 2+ словоформ, имеющих синт.\ связь
      \item Семантический~--- выражение синт.\ связи
      \item Парадигматический~--- система форм, изоморфная парадигме вершины
    \end{enumerate}
  \end{block}
\end{frame}

\begin{frame}
  \frametitle{Классификация словосочетаний}

  \begin{itemize}
    \item По частеречной принадлежности вершины \begin{itemize}
      \item Глагольные: \textit{выражать понятие}
      \item Субстантивные: \textit{частеречная принадлежность}
      \item Адъективные: \textit{очень рассеянный}
      \item Наречные: \textit{чересчур долго}
    \end{itemize}
    \item Свободные \textit{vs} несвободные~--- по наличию комплетивной связи
    \item По типу конструктивной сложности \begin{itemize}
      \item Простое: \textit{верблюжья шерсть}
      \item Сложное (куст): \textit{гладкая верблюжья шерсть}
      \item Комбинированное (дерево): \textit{пальто из верблюжьей шерсти}
    \end{itemize}
  \end{itemize}
\end{frame}

\begin{frame}
  \frametitle{Phrase}
  % TODO
\end{frame}

\subsection{Подчинительная связь}
\frame{\tableofcontents[currentsection,currentsubsection]}

\begin{frame}
  \frametitle{Подчинение}
  \framesubtitle{Субординация, гипотаксис}

  \begin{alertblock}{Подчинительная связь}
    Связь между знаменательными словами, которая позволяет выделить главный и зависимый компоненты
  \end{alertblock}

  \vfill

  \begin{itemize}
    \item Противопоставлено сочинению (также координация, паратаксис)
    \item Предопределяется грамматическими свойствами вершины
    \item Имеет типовые формальные средства выражения зависимости
    \item Имеет типовые семантические отношения между членами
  \end{itemize}
\end{frame}

\begin{frame}
  \frametitle{Типология подчинительных связей}

  \begin{enumerate}
    \item \textbf{Согласование}~--- уподобление зависимого слова главному в одноименных грамматических категориях: \textit{подчинительная связь}
    \item \textbf{Управление}~--- предопределение формы зависимого слова главным (они не обязательно совпадают): \textit{рамка валентностей}
    \item \textbf{Примыкание}~--- <<способ подчинения, который не является ни управлением, ни примыканием>> (Пешковский): \textit{устало докладывать}
  \end{enumerate}
\end{frame}

\begin{frame}
  \frametitle{Координация\textsubscript{2}}
  \framesubtitle{Одновременное согласование и управление}

  \begin{exampleblock}{Татарский}
    \textsc{GEN} у посессора, согл.\ по лицу и числу
    \begin{multicols}{2}
      \begin{itemize}
        \item \digloss
          {укытучы-нын          китаб-ы}
          {учитель\textsc{.GEN} книга\textsc{.3SG}}
          {книга учителя}
        \item \digloss
          {минем          китаб-ыv}
          {я\textsc{.GEN} книга\textsc{.1SG}}
          {моя книга}
      \end{itemize}
    \end{multicols}
  \end{exampleblock}

  \begin{exampleblock}<2->{Русский}
    \textsc{NOM} $\leftarrow$ финитность, согл.\ по лицу, числу и роду \begin{itemize}
      \item \textit{учитель читает}
      \item \textit{я читаю}
    \end{itemize}
  \end{exampleblock}
\end{frame}

\begin{frame}
  \frametitle{Сравнение согласования и управления}
  \framesubtitle{\autocite{kibrik:1992}}

  \begin{enumerate}
    \item Это способы классификации по разным основаниям
    \item Согласование, в отличие от управления, не связано с идеей подчинения
    \item Разный характер импликации: \begin{itemize}
      \item Согласование: <<если есть слова X и Y, то\ldots>>
      \item Управление: <<если есть слово X, то есть Y, который\ldots>>
    \end{itemize}
    \item В общем случае согласование~--- признак класса слов, управление~--- слова
    \item Согласование~--- явление поверхностного синтаксиса, управление имеет семантическую природу
    \item Согласование во многих языках отсутствует, управление~--- языковая универсалия
  \end{enumerate}
\end{frame}

\begin{frame}
  \frametitle{Характеристика силы связи}

  \begin{alertblock}{Сила подчинительной связи}
     Степень обязательности зависимого слова при главном
  \end{alertblock}

  \vfill

  \begin{itemize}
    \item Сильная связь $\leftarrow$ объектные и комлетивные отношения
    \item Слабая связь $\leftarrow$ атрибутивные отношения
  \end{itemize}
\end{frame}

\begin{frame}
  \frametitle{Сильные и слабые связи}

  \begin{table}
    \begin{tabularx}{.9\textwidth}{XXX}
      & Сильная связь & Слабая связь \\ \midrule \midrule
      Согласование & \textit{большой палец} & \textit{большой дом} \\ \midrule
      Управление & \textit{писать книгу} & \textit{писать ручкой} \\ \midrule
      Примыкание & \textit{стоит дорого} & \textit{стоит рубль} \\
    \end{tabularx}
  \end{table}
\end{frame}

\begin{frame}
  \frametitle{Вариативность и синонимия}

  \begin{itemize}
    \item стакан воды (N\textsubscript{2})~--- \ldots{} с водой (PP)
    \item возможность уйти (Inf)~--- \ldots{} ухода (N)~--- \ldots{} для ухода (PP)
    \item высокорослый юноша (Adj)~--- юноша высокого роста (Adj N\textsubscript{2})
  \end{itemize}
\end{frame}

\subsection{Валентность}
\frame{\tableofcontents[currentsection,currentsubsection]}

\begin{frame}
  \frametitle{Две валентности}

  \begin{alertblock}{Семантическая валентность}
    Любая переменная X, входящая в описание значения лексемы L \\
    $\sim$ Партиципант
  \end{alertblock}

  \begin{alertblock}{Синтаксическая валентность}
    Селективный признак, указывающий,
    что лексема L может иметь слово W в качестве зависимого или вершины
  \end{alertblock}

  \vfill

  \onslide<2->{
    \begin{table}
      \begin{tabularx}{.9\textwidth}{XXX}
        & Партиципант $+$ & Партиципант $-$ \\ \midrule \midrule
        Активная в. & \textit{лучше} меня & \textit{пишет} хорошо \\ \midrule
        Пассивная в. & \textit{красивый конверт} & что \textit{уходит} \\ \midrule
        & $=$ актант & $=$ сирконстант \\
      \end{tabularx}
    \end{table}
  }
\end{frame}

\begin{frame}
  \frametitle{Семантическая валентность}

  \begin{alertblock}{Валентность}
    Число актантов, которые может присоединять глагол (Теньер) \\
    $\sim$ Semantic case (Филлмор), θ-role (Хомский)
  \end{alertblock}

  \begin{alertblock}{Модель управления}
    Описание валентностей слова и способов их реализации в тексте
  \end{alertblock}

  \begin{multicols}{2}
    \begin{itemize}
      \item<2-> \textit{побеждать} \begin{enumerate}
        \item[X] `кто побеждает'~--- N\textsubscript{1}
        \item[Y] `кого побеждает'~--- N\textsubscript{4}
        \item[Z] `вид борьбы'~--- PP
      \end{enumerate}
    \item<3->
      \textit{выстрелить:} 4~валентности, 4~актанта \\
      \textit{промахнуться:} 4~валентности, 1~актант
    \end{itemize}
  \end{multicols}
\end{frame}
