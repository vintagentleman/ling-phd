\section{Предложение}

\subsection{Структурный подход}
\frame{\tableofcontents[currentsection,currentsubsection]}

\begin{frame}
  \frametitle{Предложение}

  Синтаксическая конструкция, которая является \begin{itemize}
    \item относительно независимым от контекста высказыванием,
    \item обладающим коммуникативной функцией
    \item и интонационным оформлением,
    \item включающим грамматически связанные формы слов
    \item и построенным на основе отвлеченного грамматического образца~--- структурной схемы.
  \end{itemize}
\end{frame}

\begin{frame}
  \frametitle{Предикативность}

  \begin{block}{Предикация}
    Установление связи между предложением и ситуацией
  \end{block}

  \vfill

  \begin{itemize}
    \item \textit{Лошадь белая}~--- связь устанавливается в самом предложении
    \item \textit{Белая лошадь}~--- есть пресуппозиция, что связь уже существует
  \end{itemize}
\end{frame}

\begin{frame}
  \frametitle{Парадигма предложения}

  \begin{itemize}
    \item Грамматическое значение предложения~--- предикативность, т.~е.\ комплекс модально-временных значений
    \item Предикативность существует в виде частных модальных (реальность~--- ирреальность) и временных значений
    \item Частные модально-временные значения выражаются видоизменениями формальной организации предложения, составляющими его парадигму
  \end{itemize}

  \vfill

  Полная парадигма предложения
  $=$ 3~формы изъявительного (реального) наклонения
  $+$ 5~форм ирреальных наклонений
\end{frame}

\begin{frame}
  \frametitle{Ирреальные наклонения}

  \begin{enumerate}
    \item Сослагательное~--- неосложненное значение ирреальности \\
      \textit{В доме была бы тишина}
    \item Условное~--- то же, но в фиксированном контексте \\
      \textit{Если бы в доме была тишина\ldots}
    \item Побудительное \\
      \textit{Пусть в доме будет тишина}
    \item Желательное \\
      \textit{Хоть бы в доме была тишина}
    \item Долженствовательное (разг.) \\
      \textit{В доме будь тишина!}
  \end{enumerate}
\end{frame}

\begin{frame}
  \frametitle{Структурные схемы}
  % TODO
\end{frame}
