\section{Предложение}
\frame{\tableofcontents[currentsection]}

\begin{frame}
  \frametitle{Предложение}

  Синтаксическая конструкция, которая является \begin{itemize}
    \item относительно независимым от контекста высказыванием,
    \item обладающим коммуникативной функцией
    \item и интонационным оформлением,
    \item включающим грамматически связанные формы слов
    \item и построенным на основе отвлеченного грамматического образца~--- структурной схемы.
  \end{itemize}
\end{frame}

\begin{frame}
  \frametitle{Предикативность}

  \begin{alertblock}{Предикация}
    Установление связи между предложением и ситуацией
  \end{alertblock}

  \vfill

  \begin{itemize}
    \item \textit{Лошадь белая}~--- связь устанавливается в самом предложении
    \item \textit{Белая лошадь}~--- есть пресуппозиция, что связь уже существует
  \end{itemize}
\end{frame}

\begin{frame}
  \frametitle{Полипредикативность}

  \begin{alertblock}{Сложное предложение}
    Грамматически оформленное сочетание 2+ простых предложений,
    функционирующая как сообщение о разных ситуациях или разных аспектах одной ситуации
  \end{alertblock}

  \textit{А Васька слушает да ест}~--- тоже сложное предложение: 2~предикативных центра плюс собственное значение (одновременность действий)
\end{frame}

\begin{frame}
  \frametitle{Парадигма предложения}

  \begin{itemize}
    \item Грамматическое значение предложения~--- предикативность, т.~е.\ комплекс модально-временных значений
    \item Предикативность существует в виде частных модальных (реальность~--- ирреальность) и временных значений
    \item Частные модально-временные значения выражаются видоизменениями формальной организации предложения, составляющими его парадигму
  \end{itemize}

  \vfill

  Полная парадигма предложения
  $=$ 3~формы изъявительного (реального) наклонения
  $+$ 5~форм ирреальных наклонений
\end{frame}

\begin{frame}
  \frametitle{Ирреальные наклонения}

  \begin{enumerate}
    \item Сослагательное~--- неосложненное значение ирреальности \\
      \textit{В доме была бы тишина}
    \item Условное~--- то же, но в фиксированном контексте \\
      \textit{Если бы в доме была тишина\ldots}
    \item Побудительное \\
      \textit{Пусть в доме будет тишина}
    \item Желательное \\
      \textit{Хоть бы в доме была тишина}
    \item Долженствовательное (разг.) \\
      \textit{\textsuperscript{?}В доме будь тишина!}
  \end{enumerate}
\end{frame}

\begin{frame}
  \frametitle{Структурная схема}

  Абстрактный синтаксический образец,
  по которому может быть построено отдельное минимальное относительно законченное предложение

  \vfill

  Схемы характеризуют:
  \begin{enumerate}
    \item Формальное устройство
    \item Семантика
    \item Парадигматика предложений, построенных по схеме
    \item Система регулярных реализаций
    \item Правила распространения
  \end{enumerate}
\end{frame}

\begin{frame}
  \frametitle{Схемы невопросительных предложений}
  \framesubtitle{Двухкомпонентные свободные схемы}
  \small

  \begin{enumerate}
    \item с V\textsubscript{fin} \begin{enumerate}
      \item подлежащно-сказуемостные \begin{itemize}
        \item N\textsubscript{1} V\textsubscript{fin} `отношение между субъектом и его предикативным признаком~--- действием или процессом': \textit{Лес шумит}
      \end{itemize}
      \item<2-> не подлежащно-сказуемостные \begin{itemize}
        \item V\textsubscript{3.\textsc{SG}} Inf `квалификация отвлеченно представленного действия или процесса': \textit{Следует подождать}
        \item N\textsubscript{2} [\textsc{NEG}] V\textsubscript{3.\textsc{SG}} `отношение между субъектом и его предикативным признаком~--- процессом': \textit{Несчастья не случилось}
      \end{itemize}
    \end{enumerate}

    \item<3-> без V\textsubscript{fin} \begin{enumerate}
      \item лексически неограниченные
      \item лексически ограниченные \begin{itemize}
        % \item нет N\textsubscript{2} `несуществование или отсутствие субъекта': \textit{Нет сомнений}
        \item ни N\textsubscript{2} `несуществование, отсутствие потенциально неединичного субъекта': \textit{Ни звука}
        % \item никого/ничего N\textsubscript{2}/Adj\textsubscript{2} `полное отсутствие или несуществование субъекта': \textit{Ничего нового}
        % \item никакого/ни одного/ни единого N\textsubscript{2} `полное отсутствие или несуществование субъекта': \textit{Никакого праздника}
        \item Pron\textsubscript{\textsc{NEG}} Inf `неосуществление из-за отсутствия необходимых обстоятельств' \textit{Не о чем спорить}
        \item \ldots
      \end{itemize}
    \end{enumerate}
  \end{enumerate}
\end{frame}

\begin{frame}
  \frametitle{Схемы невопросительных предложений}
  \framesubtitle{Фразеологизированные схемы}

  \begin{multicols}{2}
    \begin{itemize}
      \item с союзами \begin{itemize}
        \item N\textsubscript{1} как N\textsubscript{1}
        \item нет чтобы Inf!
      \end{itemize}
      \item с частицами \begin{itemize}
        \item вот N\textsubscript{1} так N\textsubscript{1}!
        \item Inf так Inf!
        \item ох уж эти N\textsubscript{1}!
      \end{itemize}
      \item с предлогами \begin{itemize}
        \item N\textsubscript{1} не в N\textsubscript{1}
        \item не до N\textsubscript{2}!
      \end{itemize}
      \item с местоимениями \begin{itemize}
        \item всем N\textsubscript{5} N\textsubscript{1}!
        \item чем не N\textsubscript{1}!
        \item N\textsubscript{3}-то что!
      \end{itemize}
    \end{itemize}
  \end{multicols}
\end{frame}

\begin{frame}
  \frametitle{Распространение}

  \begin{itemize}
    \item Присловные распространители: \textit{помещение} $\rightarrow$ \textit{жилое помещение}
    \item Свободный определяющий член: \textit{дом строится} $\rightarrow$ \textit{\ldots{} рабочими}
    \item Обращение: \textit{привет} $\rightarrow$ \textit{привет, мир}
  \end{itemize}
\end{frame}

\begin{frame}
  \frametitle{Модус и диктум}

  \textbf{Диктум}~--- номинативный аспект предложения, \textbf{модус}~--- коммуникативный.
  Объективная модальность $\sim$ предикативность, выражается всегда.
  Субъективная может быть выражена или нет по желанию говорящего.

  \vfill

  \begin{itemize}
    \item Отрицание
    \item Вопросительность
    \item Спец.\ лексика (напр., модальные глаголы)
  \end{itemize}
\end{frame}

\begin{frame}
  \frametitle{Пара слов о типологии}

  \begin{enumerate}
    \item Номинативный строй \begin{itemize}
      \item \textit{\textcolor{red}{Студент} проспал}
      \item \textit{\textcolor{red}{Студент} не завел \textcolor{blue}{будильник}}
    \end{itemize}

    \item<2-> Эргативный строй \begin{itemize}
      \item \textit{
        (\textcolor{cyan}{Студент} {\normalfont [фактитив]} проспал)
      }
      \item \textit{
        (\textcolor{orange}{Студент} {\normalfont [агентив]} не завел \textcolor{cyan}{будильник})
      }
      \item<3-> баск.\ \textit{
        \textcolor{orange}{Ehiztaria\underline{k}} \textcolor{cyan}{otsoa} harraptu du
      }
      `\textcolor{orange}{Охотник} поймал \textcolor{cyan}{волк\underline{а}}'
    \end{itemize}

    \item<4-> Активный строй \begin{itemize}
      \item \textit{(\textcolor{violet}{Студент} {\normalfont [актив]} проспал)}
      \item \textit{(\textcolor{olive}{Студент} {\normalfont [статив]} споткнулся)}
      \item<5-> дакота \textit{\textcolor{violet}{\underline{wa}-ti}} `я живу'~---
                \textit{\textcolor{olive}{\underline{ma}-a}} `я умираю'
    \end{itemize}
  \end{enumerate}
\end{frame}
