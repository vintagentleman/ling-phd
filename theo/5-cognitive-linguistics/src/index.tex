% \documentclass[xetex, aspectratio=169, russian]{beamer}
\usecolortheme{beaver}
\beamertemplatenavigationsymbolsempty
\setbeamertemplate{footline}[frame number]

\usepackage{fontspec}
\usepackage{polyglossia}
\usepackage[autostyle]{csquotes}
\setmainlanguage[babelshorthands]{russian}
\setotherlanguage{english}

\setmainfont{Liberation Serif}
\newfontfamily{\cyrillicfont}{Liberation Serif}
\setsansfont{Liberation Sans}
\newfontfamily{\cyrillicfontsf}{Liberation Sans}
\setmonofont{Liberation Mono}
\newfontfamily{\cyrillicfonttt}{Liberation Mono}

\usepackage{booktabs}
\usepackage{multirow}
\usepackage{makecell}
\usepackage{ltablex}
\usepackage{paralist}
\keepXColumns
\renewcommand\theadfont{\normalsize}

\usepackage{fvextra}
\usepackage{xcolor}
\usepackage{textcomp}
\usepackage{graphicx}
\usepackage{tikz}
\usepackage[edges]{forest}
\graphicspath{{../img/}}

\usepackage{mathtools}
\usepackage{amssymb}

% Библиография
\input{_biblatex}

\title[]{Автоматические диалоговые системы}

\begin{document}


\frame{\titlepage}

\section{Литература}
\frame{\tableofcontents[currentsection]}

\begin{frame}
  \frametitle{Литература}
  \nocite{*}
  \printbibliography
\end{frame}

\section{Когнитивная наука}
\frame{\tableofcontents[currentsection]}

\begin{frame}
  \frametitle{<<Первая когнитивная революция>>}
  \framesubtitle{1940--1960-е гг.}

  \begin{block}{Кризис бихевиоризма}
    \begin{itemize}
      \item Наблюдаемые процессы не сводятся к стимулам и реакциям
      \item Э.~Толмен: формула поведения включает промежуточные переменные
      \item Сознание познаваемо
    \end{itemize}
  \end{block}
\end{frame}

\begin{frame}
  \frametitle{Симпозиум по проблемам переработки информации, MIT}
  \framesubtitle{1956 г.}

  Знаковые доклады: \begin{itemize}
    \item Дж. Миллер: доклад «Магическое число семь плюс-минус два», ячеечная модель рабочей памяти человека
    \item Г. Саймон, А. Ньюэлл: модель искусственного интеллекта «Логик-теоретик», способная доказывать логические теоремы
    \item Н. Хомский: «Три модели описания языка», мысль о том, что языковые модели должны предполагать участие человека
  \end{itemize}
\end{frame}

\begin{frame}{Психолингвистика}
  \framesubtitle{1950--1960 гг. и далее}

  \begin{enumerate}
    \item Описание речевых сообщений на основе изучения механизмов порождения и восприятия речи \begin{itemize}
      \item Усвоение языка
    \end{itemize}
    \item Исследование связи между речевыми сообщениями и характеристиками участников коммуникации \begin{itemize}
      \item Афазии
    \end{itemize}
    \item Анализ речевого развития в связи с развитием личности \begin{itemize}
      \item Детская речь
    \end{itemize}
  \end{enumerate}
\end{frame}

\begin{frame}
  \frametitle{Другие теоретические источники}

  \begin{itemize}
    \item Ментализм в языкознании: В. Гумбольдт, А. Потебня, Б. Уорф
    \item Л. С. Выготский: теория речевой деятельности
    \item Лингвистический эксперимент по Л. В. Щербе
  \end{itemize}
\end{frame}

\section{Когнитивная лингвистика}
\frame{\tableofcontents[currentsection]}

\begin{frame}
  \frametitle{Вехи формального становления}

  \begin{itemize}
    \item 1976 г., книга Дж. Миллера, Ф. Джонсона-Лэрда «Язык и восприятие»
    \item 1989 г., международный лингвистический симпозиум, Дуйсбург
  \end{itemize}
\end{frame}

\begin{frame}
  \frametitle{Предмет}
  Изучение соотношения языка и сознания, роль языка в категоризации и концептуализации, связь языка с отдельными когнитивными способностями

  \vfill

  \begin{enumerate}
    \item Категоризация~--- упорядочение полученных знаний, распределение их по рубрикам, существующим в сознании человека
    \item Концептуализация~--- определение набора когнитивных признаков, которые позволяют человеку оперировать с понятием и представлением о явлении и отличать его от других
  \end{enumerate}
\end{frame}

\begin{frame}
  \frametitle{Когнитивные обязательства}

  \begin{block}{Обязательство обобщения}
    Поиск общих принципов и закономерностей, действующих на разных языковых уровнях
  \end{block}

  \begin{block}{Когнитивное обязательство}
    Лингвистические построения не должны противоречить идеям других когнитивных наук
  \end{block}
\end{frame}

\begin{frame}
  \frametitle{<<Когнитивный шестиугольник>>}

  \begin{center}
    \begin{tikzpicture}
      \newdimen\R
      \R=2.7cm
      \draw (0:\R) \foreach \x in {60,120,...,360} {  -- (\x:\R) };
      \foreach \x/\l/\p in
      { 60/{Лингвистика}/above,
        120/{Философия}/above,
        180/{Психология}/left,
        240/{ИИ}/below,
        300/{Нейронаука}/below,
        360/{Антропология}/right
      }
      \node[inner sep=1pt,circle,draw,fill,label={\p:\l}] at (\x:\R) {};
    \end{tikzpicture}
  \end{center}
\end{frame}

\section{Известные направления}
\frame{\tableofcontents[currentsection]}

\begin{frame}
  \frametitle{Теория концептуальной метафоры (Лакофф, Джонсон))}


\end{frame}

\begin{frame}
  \frametitle{Естественный семантический метаязык А. Вежбицкой}


\end{frame}

\begin{frame}
  \frametitle{Модель Р. Шенка}


\end{frame}
