% \documentclass[xetex, aspectratio=169, russian]{beamer}
\beamertemplatenavigationsymbolsempty
\setbeamertemplate{footline}[frame number]

\usepackage{fontspec}
\usepackage{polyglossia}
\usepackage[autostyle]{csquotes}
\setmainlanguage[babelshorthands]{russian}
\setotherlanguage{english}

\setmainfont{CMU Serif}
\setsansfont{CMU Sans Serif}
\setmonofont{CMU Typewriter Text}

\usepackage{booktabs}
\usepackage{multirow}
\usepackage{makecell}
\usepackage{ltablex}
\usepackage{paralist}
\keepXColumns
\renewcommand\theadfont{\normalsize}

\usepackage{fvextra}
\usepackage{xcolor}
\usepackage{textcomp}
\usepackage{graphicx}
\usepackage[edges]{forest}
\graphicspath{{../img/}}

\usepackage{mathtools}
\usepackage{amssymb}

\usepackage{covington}
\renewcommand*\glosslinetrans[1]{`#1'}

% Библиография
\usepackage[
    backend=biber,
    bibencoding=utf8,
    style=gost-authoryear,
    language=auto,
    autolang=other,
    clearlang=true,
    sortcites=true,
    movenames=false,
    minbibnames=3,
    maxbibnames=5
]{biblatex}

% Сортировка библиографии
\DeclareSourcemap{
    \maps[datatype=bibtex]{
        \map{
            \step[fieldsource=langid, match=russian, final]
            \step[fieldset=presort, fieldvalue={a}]
        }
        \map{
            \step[fieldsource=langid, notmatch=russian, final]
            \step[fieldset=presort, fieldvalue={z}]
        }
    }
}

% Убираем неразрывные пробелы перед двоеточием и точкой с запятой
\makeatletter

\renewcommand*{\addcolondelim}{%
    \begingroup%
    \def\abx@colon{%
        \ifdim\lastkern>\z@\unkern\fi%
        \abx@puncthook{:}\space}%
    \addcolon%
    \endgroup%
}

\renewcommand*{\addsemicolondelim}{%
    \begingroup%
    \def\abx@semicolon{%
        \ifdim\lastkern>\z@\unkern\fi%
        \abx@puncthook{;}\space}%
    \addsemicolon%
    \endgroup%
}

\makeatother

% Правка записей типа thesis, чтобы дважды не писался автор
\DeclareBibliographyDriver{thesis}{%
    \usebibmacro{bibindex}%
    \usebibmacro{begentry}%
    \usebibmacro{heading}%
    \newunit
    \usebibmacro{author}%
    \setunit*{\labelnamepunct}%
    \usebibmacro{thesistitle}%
    \setunit{\respdelim}%
    \newunit\newblock
    \printlist[semicolondelim]{specdata}%
    \newunit
    \usebibmacro{institution+location+date}%
    \newunit\newblock
    \usebibmacro{chapter+pages}%
    \newunit
    \printfield{pagetotal}%
    \newunit\newblock
    \usebibmacro{doi+eprint+url+note}%
    \newunit\newblock
    \usebibmacro{addendum+pubstate}%
    \setunit{\bibpagerefpunct}\newblock
    \usebibmacro{pageref}%
    \newunit\newblock
    \usebibmacro{related:init}%
    \usebibmacro{related}%
    \usebibmacro{finentry}%
}

% Короткое тире в интервалах страниц
\DefineBibliographyExtras{russian}{\protected\def\bibrangedash{\textendash}}

% Счётчик цитируемых источников
\usepackage{totcount}
\newtotcounter{citnum}
\AtEveryBibitem{\stepcounter{citnum}}

% Источники
\addbibresource{refs.bib}


% https://tex.stackexchange.com/a/30726
\newcommand\blfootnote[1]{%
  \begingroup
  \renewcommand\thefootnote{}\footnote{#1}%
  \addtocounter{footnote}{-1}%
  \endgroup
}

\title[]{Синтаксис}

\begin{document}


\frame{\titlepage}

\section{Литература}
\frame{\tableofcontents[currentsection]}

\begin{frame}
  \frametitle{Литература}
  \nocite{*}
  \printbibliography
\end{frame}

\section{Прагматика}

\begin{frame}
  \frametitle{Разделы лингвистики}

  \begin{center}
    \vfill
    \begin{forest}
      [Лингвистическая прагматика
        [Speech act theory]
        [Linguistic pragmatics]
      ]
    \end{forest}

    \vfill
    $+$ Анализ дискурса
    \vfill
  \end{center}
\end{frame}

\subsection{Теория речевых актов}
\frame{\tableofcontents[currentsection,currentsubsection]}

\begin{frame}
  \frametitle{Теория речевых актов (Остин)}

  \begin{exampleblock}{Речевой акт (РА)}
    Акт произнесения говорящим некоторого высказывания, построенного по правилам языкового кода,
    обладающего пропозициональным содержанием и ориентированной на адресата иллокутивной функцией
  \end{exampleblock}

  \vfill

  Участники РА~--- \begin{enumerate}
    \item носители общих социальных ролей,
    \item обладатели общей речевой компетенцией
    \item и общего фонда знаний и представлений о мире.
  \end{enumerate}
\end{frame}

\begin{frame}
  \frametitle{Соответствие предложениям}

  1 предложение~--- несколько типов РА. \textit{я поговорю с твоими родителями:} \begin{itemize}
    \item сообщение
    \item предупреждение
    \item принятие обязательства
  \end{itemize}

  \vfill

  1 тип РА~--- несколько синт.\ классов. Реквестив~--- \begin{itemize}
    \item вопросительное: \textit{ты можешь сегодня зайти?}
    \item повествовательное: \textit{хочу узнать, не можешь ли ты сегодня зайти}
  \end{itemize}
\end{frame}

\begin{frame}
  \frametitle{Аспекты РА}

  \begin{itemize}
    \item Локуция~--- построение и произнесение высказывания
    \item \alert<2->{Иллокуция}~--- коммуникативное намерение говорящего
    \item Перлокуция~-- воздействие на адресата (не входит в структуру РА)
  \end{itemize}
\end{frame}

\begin{frame}
  \frametitle{Классификация РА (Серль)}
  \framesubtitle{Типы иллокутивных функций}

  \begin{itemize}
    \item Репрезентативы (ассертивы): \textit{поезд пришел}. Сообщение, утверждение, констатация, описание, объяснение
    \item Директивы (прескриптивы): \textit{пошел вон!}, \textit{который час?}. Приказ, требование, распоряжение, просьба, мольба
    \item Комиссивы: \textit{обещаю не опаздывать}. Обещание, клятва, гарантия, предупреждение, угроза
    \item Экспрессивы: \textit{извините за беспокойство}. Благодарность, упрек, поздравление, формулы этикета
    \item Декларативы (вердиктивы): \textit{объявляю вас мужем и женой}. Изменение мира посредством речевого акта
  \end{itemize}
\end{frame}

\begin{frame}
  \frametitle{Перформативы}

  \textit{объявляю вас мужем и женой}~--- РА эквивалентен действию.

  Оценка не по истинности/ложности, а по успешности.

  \vfill

  Критерии успешности: \begin{itemize}
    \item наличие соответствующих полномочий
    \item выполнимость перформативного акта
    \item способность адресата его выполнить
  \end{itemize}
\end{frame}

\subsection{Лингвистическая прагматика}
\frame{\tableofcontents[currentsection,currentsubsection]}

\begin{frame}
  \frametitle{Лингвистическая прагматика}

  \begin{block}{Принцип кооперации (Грайс)}
    <<Твой коммуникативный вклад на данном шаге диалога должен быть таким,
    какого требует совместно принятая цель (направление) этого диалога>>.
  \end{block}
\end{frame}

\begin{frame}
  \frametitle{Постулаты речевого общения}

  \begin{itemize}
    \item Количества (объема передаваемой информации)
    \item Качества~--- о том, что следует придерживаться истины
    \item Отношения (релевантности)
    \item Способа~--- о формальной стороне общения
  \end{itemize}
\end{frame}

\begin{frame}
  \frametitle{Критика принципа кооперации}

  \begin{itemize}
    \item Размытость постулатов
    \item Идеальная модель диалога в атмосфере сотрудничества
    \item Неуниверсальность постулатов в разных культурах
  \end{itemize}
\end{frame}

\begin{frame}
  \frametitle{Скрытые смыслы}

  \begin{exampleblock}{Импликатура}
    Те компоненты смысла, существование которых необходимо предположить,
    чтобы сохранить презумпцию принципа кооперации
  \end{exampleblock}

  \vfill

  Выявить импликатуры позволяют нарушения постулатов: \begin{enumerate}
    \item \textit{Ты вымыл посуду и поставил на место?~--- Я вымыл посуду}
    \item \textit{Вы готовы к экзамену?~--- Да}
    \item \textit{Ты считаешь ее красивой?~--- Она хорошо одевается}
    \item \textit{Не могли бы вы передать соль?} (вм.\ \textit{Передайте соль})
  \end{enumerate}
\end{frame}

\begin{frame}
  \frametitle{Круг вопросов прагматики}
  \framesubtitle{В связи с субъектом речи}

  \begin{enumerate}
    \item Иллокутивные силы
    \item Правила речевого общения
    \item Речевая тактика, типы речевого поведения
    \item Прагматическое значение: косвенные смысли, намеки
    \item Референция говорящего
    \item Прагматические пресуппозиции
    \item \ldots
  \end{enumerate}
\end{frame}

\begin{frame}
  \frametitle{Круг вопросов прагматики}

  \begin{block}{В связи с адресатом речи}
    \begin{itemize}
      \item Правила интерпретации речи
      \item Перлокутивный эффект
      \item Типы речевого реагирования
    \end{itemize}
  \end{block}

  \begin{block}{В связи с ситуацией общения}
    \begin{itemize}
      \item Дейксис
      \item Влияние ситуации на форму и содержание коммуникации
    \end{itemize}
  \end{block}
\end{frame}

\section{Дихотомии}
\frame{\tableofcontents[currentsection]}

\begin{frame}
  \frametitle{Устная и письменная речь}

  \begin{itemize}
    \item Устная речь~--- первичная форма существования языка как в онто-, так и в филогенезе
    \item Письменная традиция европоцентричной мысли повлияла на то, что она долгое время не изучалась per se
  \end{itemize}
\end{frame}

\begin{frame}
  \frametitle{Отличия устного и письменного модусов}
  \framesubtitle{Чейф}

  \begin{enumerate}
    \item Временной режим: при устном дискурсе порождение речи и понимание происходят синхронно \begin{itemize}
      \item[$\Rightarrow$] Устная речь фрагментирована, порождается отдельными квантами
      \item[$\Rightarrow$] Письменной речи свойственна интеграция предикаций
    \end{itemize}
    \item<2-> В письменном дискурсе типично нет контакта между говорящим и адресатом во времени и пространстве \begin{itemize}
      \item[$\Rightarrow$] Шифтеры, указания на мысли и эмоции, жесты
      \item[$\Rightarrow$] Отстранение коммуникантов от сообщаемого
    \end{itemize}
  \end{enumerate}
\end{frame}

\begin{frame}
  \frametitle{Отличия устного и письменного модусов}
  \framesubtitle{Brown, Yule}

  Синтаксис устного дискурса: \begin{itemize}
    \item Неполные предложения, без подчинения, с преобладанием активных форм глаголов
    \item Частое отсутствие маркирования внутритекстовых связей
    \item Меньшая семантическая <<плотность>>, отсутствие сложных ИГ
    \item Topic~--- comment vs Subject~--- object
    \item Повтор одной и той же синтаксической структуры
  \end{itemize}

  \vfill

  Также автокоррекция, особенности лексикона (слова-заполнители \textit{well, I think;} слова с абстрактным значением \textit{thing, stuff)}
\end{frame}

\begin{frame}
  \frametitle{Квантитативная оценка (Чувилина)}

  \begin{table}
    \small
    \begin{tabularx}{\textwidth}{p{5cm}XX}
      \toprule
      & Устный дискурс & Письменный дискурс \\ \midrule\midrule
      Число предикаций (финитных и деепричастных) & 45 & 27 \\ \midrule
      Среднее число слов на предикацию & 6,9 & 5,4 \\ \midrule
      Среднее число предикатных слов на предикацию & 1,2 & 1,7 \\ \midrule
      Число дискурсивных маркеров & 6 & 0 \\ \bottomrule
    \end{tabularx}
  \end{table}
\end{frame}

\begin{frame}
  \frametitle{Монолог и диалог}

  \begin{table}
  \small
  \begin{tabularx}{.9\textwidth}{XX}
    \toprule
    Монолог & Диалог \\ \midrule\midrule
    Пассивное восприятие & Роль адресата активна \\ \midrule
    Интраперсональный РА & Интерперсональный РА \\ \midrule
    Композиционная завершенность & Спонтанность \\ \bottomrule
    \end{tabularx}
  \end{table}

  Винокур: нет непроходимых границ, любой монолог в той или иной степени диалогизирован
\end{frame}

\begin{frame}
  \frametitle{Вербальные и невербальные средства общения}

  \begin{block}{Паралингвистика}
    Раздел лингвистики, изучающий невербальные \textbf{(неязыковые)} средства речевого общения и передающие смысловую информацию
  \end{block}

  \vfill

  Виды: \begin{itemize}
    \item Фонационные: FIII+, темп, громкость, заполнители пауз
    \item Кинетические: жестикуляция, поза, мимика
    \item Графические: особенности почерка, употребление букв
  \end{itemize}
\end{frame}

\begin{frame}
  \frametitle{Функции паралингвистических единиц}

  \begin{itemize}
    \item Сообщает дополнительную информацию, в~т.\,ч.\ противоположную
    \item Замещает пропущенные вербальные компоненты
    \item Дублирует информацию, сообщаемую вербально (избыточность)
  \end{itemize}
\end{frame}
